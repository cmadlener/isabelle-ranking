\documentclass{beamer}
\usetheme[sectionpage=none]{metropolis}

\usepackage{isabelle,isabellesym}
\newcommand*{\term}[1]{{\isaspacing\isastyle \input{#1}}}
\newcommand*{\Term}[1]{{\isaspacing\isastyle #1}}
\newcommand*{\mTerm}[1]{\text{\Term{#1}}}

\usepackage{mathtools}

\DeclarePairedDelimiter\card{\lvert}{\rvert}

\usepackage[ruled]{algorithm2e}
\resetcounteronoverlays{algocf}


\SetKwInput{Init}{Initialization}
\SetKwInput{Online}{Online Matching}
\SetKwFor{Arrival}{On arrival of}{}{}
\SetKwIF{If}{ElseIf}{Else}{if}{}{else if}{else}{}

\title{Formal Verification of the RANKING algorithm for Online Bipartite Matching}
\author{Christoph Madlener}
\date{22.06.2022}

\begin{document}
\begin{frame}[plain]
  \titlepage
\end{frame}

\section{Introduction}
\begin{frame}
  \frametitle{Online Bipartite Matching (OBM)}
  \begin{columns}
    \begin{column}{.65\textwidth}
      \begin{exampleblock}{Input}
        \begin{itemize}[<+->]
          \item \emph{bipartite} graph $G = (U,V,E)$
          \item \emph{offline} vertices $V$ are known
          \item \emph{online} vertices $U$ reveal edges on arrival
        \end{itemize}
      \end{exampleblock}
    \end{column}
    \begin{column}{.3\textwidth}
      \only<1>{\includegraphics[scale=0.7]{figures/graph_complete}}%
      \only<2-3>{\includegraphics[scale=0.7]{figures/graph_offline_only}}%
      \only<4>{\includegraphics[scale=0.7]{figures/graph_arrival_1}}%
      \only<5>{\includegraphics[scale=0.7]{figures/graph_arrival_2}}%
      \only<6>{\includegraphics[scale=0.7]{figures/graph_arrival_3}}%
      \only<7>{\includegraphics[scale=0.7]{figures/graph_arrival_4}}%
      \only<8-9>{\includegraphics[scale=0.7]{figures/graph_matching_1}}%
      \only<10>{\includegraphics[scale=0.7]{figures/graph_matching_2}}%
      \only<11>{\includegraphics[scale=0.7]{figures/graph_matching_3}}%
      \only<12>{\includegraphics[scale=0.7]{figures/graph_matching_4}}%
      \only<13>{\includegraphics[scale=0.7]{figures/graph_matching_5}}%
      \only<14>{\includegraphics[scale=0.7]{figures/graph_matching_6}}%
      \only<15>{\includegraphics[scale=0.7]{figures/graph_matching_7}}%
      \only<16>{\includegraphics[scale=0.7]{figures/graph_matching_8}}%
    \end{column}
  \end{columns}
  \onslide<8->
  \begin{alertblock}{Task}
    \begin{itemize}
      \item<8-> on arrival of $u \in U$, match to \emph{unmatched} neighbor $v \in V$ (or not)
      \item<9-> maximize size of resulting matching
    \end{itemize}   
  \end{alertblock}
\end{frame}

\begin{frame}
  \frametitle{Competitive Ratio}
  \begin{columns}
    \begin{column}{.65\textwidth}
      \begin{alertblock}{Performance of online algorithm $\mathcal{A}$}
        \begin{itemize}
          \item Compare $\mathcal{A}$ to best offline algorithm
        \end{itemize}
      \end{alertblock}
    \end{column}
    \begin{column}{.3\textwidth}
      \only<1>{\includegraphics[scale=0.7]{figures/graph_matching_8}}%
      \only<2->{\includegraphics[scale=0.7]{figures/graph_comp_ratio}}
    \end{column}
  \end{columns}
  \onslide<3->
  \begin{block}{Competitive ratio for OBM}
    \only<-3>{\[
      \min_G \min_\pi \frac{\card{\mathcal{A}(G,\pi)}}{\card{M}}
    \]}
    \only<4->{\[
      \min_G \min_\pi \frac{\mathbb{E}\big[\card{\mathcal{A}(G,\pi)}\big]}{\card{M}}
    \]}
    where $M$ is a maximum cardinality matching in $G$.
  \end{block}
\end{frame}

\begin{frame}
  \frametitle{RANKING}
  \onslide<+->
  \begin{itemize}
    \item simple randomized algorithm due to Karp, Vazirani, and Vazirani is optimal~\cite{karp1990}
  \end{itemize}
  \onslide<+->
  \begin{algorithm}[H]
  \small
  \DontPrintSemicolon
  \caption{RANKING}\label{alg:ranking}
  \Init{Choose a random permutation (ranking) $\sigma$ of $V$}
  \Online{}
  \Arrival{$u \in U$}{
    $N(u) \gets \text{set of unmatched neighbors of }u$\\
    \If{$N(u) \neq \emptyset$}{
      match $u$ to the vertex $v \in N(u)$ that minimizes $\sigma(v)$
    }
  }
  \end{algorithm}
  \onslide<+->
  \begin{itemize}
    \item competitive ratio of $1 - \frac{1}{e}$ (best possible)
  \end{itemize}
\end{frame}

\begin{frame}
  \frametitle{Formalization Outline}
  \begin{itemize}[<+->]
    \item formalization follows proof due to Birnbaum, and Mathieu~\cite{birnbaum2008}
    \item two \emph{sections}
    \begin{enumerate}
      \item \only<3-4>{Combinatorics} \only<5->{\textbf{Combinatorics}}
      \item Probability theory
    \end{enumerate}
    \item Most involved part: \emph{Lemma 2}
    
    \emph{"when removing a vertex $x$ from the graph, then the runs on the original
      graph, and the one without $x$, differ by at most one alternating path, starting
      at $x$"}
  \end{itemize}
\end{frame}

\section{Combinatorics}
\begin{frame}
  \frametitle{A "simple structural observation"}
  \begin{columns}
    \begin{column}{.3\textwidth}
      \only<1>{\includegraphics[scale=1]{figures/ranking_before}}%
      \only<2->{\includegraphics[scale=1]{figures/ranking_after}}
    \end{column}
    \begin{column}{.65\textwidth}
      \onslide<3->
      \begin{alertblock}{Keys to formal proof}
        \begin{itemize}
          \item<3-> \emph{non-recursive} specification of matching on $G$ with arrival order $\pi$,
          and ranking $\sigma$
          \item<4-> gives interchangebility of offline and online vertices
          \item<5-> full specification of path with mutually recursive functions \emph{zig} and \emph{zag}
          \item<6-> Berge's Lemma formalized by Abdulaziz~\cite{abdulaziz2019}
        \end{itemize}
      \end{alertblock}
    \end{column}
  \end{columns}
  
  \note[itemize]{
    \item Berge's Lemma not required for Lemma 2, but for consequence
  }
\end{frame}

%\begin{frame}
%  \frametitle{Remaining combinatorics}
%\end{frame}

\section{Randomization}
\begin{frame}
  \frametitle{Randomization}
  \begin{itemize}[<+->]
    \item rephrase everything as \emph{\_ pmf} (probability mass function)
    \item finite probability spaces over permutations \textrightarrow{} lots of sums
  \end{itemize}
  \onslide<+->
  \begin{alertblock}{Switching probability spaces}
    \begin{itemize}[<+->]
      \item choosing a random permutation \textbf{vs.}
      
      choosing a random permutation, a random vertex, and putting that vertex at index $t$
      \item choosing a random permutation of the original offline vertices \textbf{vs.}
      
      choosing a random permutation of the \emph{reduced} offline vertices
    \end{itemize}
  \end{alertblock}
\end{frame}

\begin{frame}
  \frametitle{References}
  \bibliographystyle{abbrv}
  \bibliography{../document/root}
\end{frame}

\end{document}

